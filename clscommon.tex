\usetheme[progressbar=frametitle,block=fill]{metropolis}
\setbeamertemplate{theorems}[numbered]
\RequirePackage{appendixnumberbeamer}

\RequirePackage{xcolor}
\RequirePackage{algorithm}
\RequirePackage{algpseudocode}
\RequirePackage{amsfonts}
\RequirePackage{amsmath}
\RequirePackage{amssymb}
\RequirePackage{amsthm}
\RequirePackage{bm}
\RequirePackage{bigdelim}
\RequirePackage{bigstrut}
\RequirePackage{bookmark}
\RequirePackage{booktabs}
\RequirePackage[scale=2]{ccicons}
\RequirePackage{cprotect}
\RequirePackage[UTF8]{ctex}
\RequirePackage{etoolbox}
\RequirePackage{extpfeil}
\RequirePackage{fancyhdr}
\RequirePackage{hyperref}
\RequirePackage{ifthen}
\RequirePackage{subcaption}
\RequirePackage{tabularx}
\RequirePackage{tikz}
\RequirePackage{url}
\RequirePackage{xspace}
\RequirePackage{longtable}
\RequirePackage{mathtools}
\RequirePackage{multirow}
\RequirePackage{multicol}
\RequirePackage{geometry}
\RequirePackage{xfp}

\RequirePackage{pgfplots}
\usepgfplotslibrary{dateplot}

\usetikzlibrary{calc}
\usetikzlibrary{math}
\usetikzlibrary{intersections}


% Font setting
\RequirePackage{eulervm}
\RequirePackage{fontspec}
\setsansfont{Fira Sans}
\setmonofont{Fira Code}[Contextuals=Alternate]


% Image
\newcommand{\includeimage}[3][scale=1]{
	\begin{figure}
		\centering
		\includegraphics[#1]{image/#2}
		\caption{#3}
		\label{fig:#2}
	\end{figure}
}

\newcommand{\includetikzimage}[2]{
	\begin{figure}
		\centering
		\input{image/#1}
		\caption{#2}
		\label{fig:#1}
	\end{figure}
}


% Code style
\RequirePackage{listings}
\RequirePackage{lstfiracode}
\lstdefinestyle{common}{
	style=FiraCodeStyle,
	belowcaptionskip=1\baselineskip,
	breaklines=true,
	xleftmargin=\parindent,
	showstringspaces=true,
	numbers=left,
	numberstyle=\ttfamily\small,
	basicstyle=\ttfamily,
	stepnumber=1,
	frame=single
}

\lstdefinestyle{code}{
	style=common,
	keywordstyle=\bfseries\color{green!40!black},
	commentstyle=\itshape\color{red!80!black},
	identifierstyle=\color{blue},
	stringstyle=\color{purple!40!black}
}

\lstdefinestyle{c++}{
	style=code,
	language=C++
}

\lstdefinestyle{c}{
	style=code,
	language=C
}

\lstdefinestyle{py}{
	style=code,
	language=Python
}

\lstdefinestyle{java}{
	style=code,
	language=Java
}

\newcommand{\includecode}[2][c++]{
	\vspace{0.3cm}
	\lstset{style=#1}
	\lstinputlisting[label={code:#2}]{code/#2}
	\vspace{0.3cm}
}

\lstdefinestyle{pascal}{
	style=code,
	language=Pascal
}


% Bib
\newcommand{\listofbib}[1][references]{
	\nocite{*}
	\bibliographystyle{plain}
	\bibliography{bib/#1}
}


% Throrem
\renewcommand\qedsymbol{$\blacksquare$}


% User defined math command
\newcommand{\lcm}{\operatorname{lcm}}
\newcommand{\nequiv}{~{\equiv}\llap{/\,}~}
\newcommand{\subjectto}{~s.t.~}


% Others
\newcommand{\forcenewline}{\leavevmode\newline}
\newcommand{\questions}{
	\begin{frame}[standout]
		Questions?
	\end{frame}
}


% [verbose]{xl}{yl}{r}{x}{y}{\theta}
\newcommand{\drawcard}[7][normal]{
	\def\vxl{#2}
	\def\vyl{#3}
	\def\vr{#4}
	\def\vx{#5}
	\def\vy{#6}
	\def\vdeg{#7/pi*180}
	\def\vcd{cos(#7)}
	\def\vsd{sin(#7)}


	\def\vxcx{\vx}
	\def\vycx{\vy}
	\coordinate (cx) at (\vxcx,\vycx);


	\def\vxcaa{\fpeval{(\vxl/2-\vr)*\vcd-(\vyl/2-\vr)*\vsd+\vx+\vr}}
	\def\vycaa{\fpeval{(\vxl/2-\vr)*\vsd+(\vyl/2-\vr)*\vcd+\vy}}
	\coordinate (c00) at (\vxcaa,\vycaa);

	\def\vxcba{\fpeval{-(\vxl/2-\vr)*\vcd-(\vyl/2-\vr)*\vsd+\vx}}
	\def\vycba{\fpeval{-(\vxl/2-\vr)*\vsd+(\vyl/2-\vr)*\vcd+\vy+\vr}}
	\coordinate (c10) at (\vxcba,\vycba);

	\def\vxcca{\fpeval{-(\vxl/2-\vr)*\vcd+(\vyl/2-\vr)*\vsd+\vx-\vr}}
	\def\vycca{\fpeval{-(\vxl/2-\vr)*\vsd-(\vyl/2-\vr)*\vcd+\vy}}
	\coordinate (c20) at (\vxcca,\vycca);

	\def\vxcda{\fpeval{(\vxl/2-\vr)*\vcd+(\vyl/2-\vr)*\vsd+\vx}}
	\def\vycda{\fpeval{(\vxl/2-\vr)*\vsd-(\vyl/2-\vr)*\vcd+\vy-\vr}}
	\coordinate (c30) at (\vxcda,\vycda);


	\def\vxca{\fpeval{(\vxl/2-\vr)*\vcd-(\vyl/2-\vr)*\vsd+\vx}}
	\def\vyca{\fpeval{(\vxl/2-\vr)*\vsd+(\vyl/2-\vr)*\vcd+\vy}}
	\coordinate (c0) at (\vxca,\vyca);

	\def\vxc1{\fpeval{-(\vxl/2-\vr)*\vcd-(\vyl/2-\vr)*\vsd+\vx}}
	\def\vyc1{\fpeval{-(\vxl/2-\vr)*\vsd+(\vyl/2-\vr)*\vcd+\vy}}
	\coordinate (c1) at (\vxc1,\vyc1);

	\def\vxcc{\fpeval{-(\vxl/2-\vr)*\vcd+(\vyl/2-\vr)*\vsd+\vx}}
	\def\vycc{\fpeval{-(\vxl/2-\vr)*\vsd-(\vyl/2-\vr)*\vcd+\vy}}
	\coordinate (c2) at (\vxcc,\vycc);

	\def\vxcd{\fpeval{(\vxl/2-\vr)*\vcd+(\vyl/2-\vr)*\vsd+\vx}}
	\def\vycd{\fpeval{(\vxl/2-\vr)*\vsd-(\vyl/2-\vr)*\vcd+\vy}}
	\coordinate (c3) at (\vxcd,\vycd);


	\def\vxcab{\fpeval{(\vxl/2-\vr)*\vcd-(\vyl/2-\vr)*\vsd+\vx}}
	\def\vycab{\fpeval{(\vxl/2-\vr)*\vsd+(\vyl/2-\vr)*\vcd+\vy+\vr}}
	\coordinate (c01) at (\vxcab,\vycab);

	\def\vxcbb{\fpeval{-(\vxl/2-\vr)*\vcd-(\vyl/2-\vr)*\vsd+\vx-\vr}}
	\def\vycbb{\fpeval{-(\vxl/2-\vr)*\vsd+(\vyl/2-\vr)*\vcd+\vy}}
	\coordinate (c11) at (\vxcbb,\vycbb);

	\def\vxccb{\fpeval{-(\vxl/2-\vr)*\vcd+(\vyl/2-\vr)*\vsd+\vx}}
	\def\vyccb{\fpeval{-(\vxl/2-\vr)*\vsd-(\vyl/2-\vr)*\vcd+\vy-\vr}}
	\coordinate (c21) at (\vxccb,\vyccb);

	\def\vxcdb{\fpeval{(\vxl/2-\vr)*\vcd+(\vyl/2-\vr)*\vsd+\vx+\vr}}
	\def\vycdb{\fpeval{(\vxl/2-\vr)*\vsd-(\vyl/2-\vr)*\vcd+\vy}}
	\coordinate (c31) at (\vxcdb,\vycdb);


	\def\vxraa{\fpeval{(\vxl/2)*\vcd-(\vyl/2-\vr)*\vsd+\vx}}
	\def\vyraa{\fpeval{(\vxl/2)*\vsd+(\vyl/2-\vr)*\vcd+\vy}}
	\coordinate (r00) at (\vxraa,\vyraa);

	\def\vxrba{\fpeval{-(\vxl/2-\vr)*\vcd-(\vyl/2)*\vsd+\vx}}
	\def\vyrba{\fpeval{-(\vxl/2-\vr)*\vsd+(\vyl/2)*\vcd+\vy}}
	\coordinate (r10) at (\vxrba,\vyrba);

	\def\vxrca{\fpeval{-(\vxl/2)*\vcd+(\vyl/2-\vr)*\vsd+\vx}}
	\def\vyrca{\fpeval{-(\vxl/2)*\vsd-(\vyl/2-\vr)*\vcd+\vy}}
	\coordinate (r20) at (\vxrca,\vyrca);

	\def\vxrda{\fpeval{(\vxl/2-\vr)*\vcd+(\vyl/2)*\vsd+\vx}}
	\def\vyrda{\fpeval{(\vxl/2-\vr)*\vsd-(\vyl/2)*\vcd+\vy}}
	\coordinate (r30) at (\vxrda,\vyrda);


	\def\vxra{\fpeval{(\vxl/2)*\vcd-(\vyl/2)*\vsd+\vx}}
	\def\vyra{\fpeval{(\vxl/2)*\vsd+(\vyl/2)*\vcd+\vy}}
	\coordinate (r0) at (\vxra,\vyra);

	\def\vxr1{\fpeval{-(\vxl/2)*\vcd-(\vyl/2)*\vsd+\vx}}
	\def\vyr1{\fpeval{-(\vxl/2)*\vsd+(\vyl/2)*\vcd+\vy}}
	\coordinate (r1) at (\vxr1,\vyr1);

	\def\vxrc{\fpeval{-(\vxl/2)*\vcd+(\vyl/2)*\vsd+\vx}}
	\def\vyrc{\fpeval{-(\vxl/2)*\vsd-(\vyl/2)*\vcd+\vy}}
	\coordinate (r2) at (\vxrc,\vyrc);

	\def\vxrd{\fpeval{(\vxl/2)*\vcd+(\vyl/2)*\vsd+\vx}}
	\def\vyrd{\fpeval{(\vxl/2)*\vsd-(\vyl/2)*\vcd+\vy}}
	\coordinate (r3) at (\vxrd,\vyrd);


	\def\vxrab{\fpeval{(\vxl/2-\vr)*\vcd-(\vyl/2)*\vsd+\vx}}
	\def\vyrab{\fpeval{(\vxl/2-\vr)*\vsd+(\vyl/2)*\vcd+\vy}}
	\coordinate (r01) at (\vxrab,\vyrab);

	\def\vxrbb{\fpeval{-(\vxl/2)*\vcd-(\vyl/2-\vr)*\vsd+\vx}}
	\def\vyrbb{\fpeval{-(\vxl/2)*\vsd+(\vyl/2-\vr)*\vcd+\vy}}
	\coordinate (r11) at (\vxrbb,\vyrbb);

	\def\vxrcb{\fpeval{-(\vxl/2-\vr)*\vcd+(\vyl/2)*\vsd+\vx}}
	\def\vyrcb{\fpeval{-(\vxl/2-\vr)*\vsd-(\vyl/2)*\vcd+\vy}}
	\coordinate (r21) at (\vxrcb,\vyrcb);

	\def\vxrdb{\fpeval{(\vxl/2)*\vcd+(\vyl/2-\vr)*\vsd+\vx}}
	\def\vyrdb{\fpeval{(\vxl/2)*\vsd-(\vyl/2-\vr)*\vcd+\vy}}
	\coordinate (r31) at (\vxrdb,\vyrdb);


	\ifstrequal{#1}{verbose}{%
		\draw[very thin,dashed] (c0) circle[radius=\vr];
		\draw[very thin,dashed] (c1) circle[radius=\vr];
		\draw[very thin,dashed] (c2) circle[radius=\vr];
		\draw[very thin,dashed] (c3) circle[radius=\vr];

		% \draw[very thin,dashed, draw=yellow!50!black] (c0) -- (c00);
		% \draw[very thin,dashed, draw=yellow!50!black] (c0) -- (c01);
		% \draw[very thin,dashed, draw=yellow!50!black] (c1) -- (c10);
		% \draw[very thin,dashed, draw=yellow!50!black] (c1) -- (c11);
		% \draw[very thin,dashed, draw=yellow!50!black] (c2) -- (c20);
		% \draw[very thin,dashed, draw=yellow!50!black] (c2) -- (c21);
		% \draw[very thin,dashed, draw=yellow!50!black] (c3) -- (c30);
		% \draw[very thin,dashed, draw=yellow!50!black] (c3) -- (c31);

		\draw[very thin,dashed] (c0) -- (r00);
		\draw[very thin,dashed] (c0) -- (r01);
		\draw[very thin,dashed] (c1) -- (r10);
		\draw[very thin,dashed] (c1) -- (r11);
		\draw[very thin,dashed] (c2) -- (r20);
		\draw[very thin,dashed] (c2) -- (r21);
		\draw[very thin,dashed] (c3) -- (r30);
		\draw[very thin,dashed] (c3) -- (r31);

		\filldraw[fill=black] (cx) circle[radius=.05];
		\filldraw[very thin,draw=red!30!black,fill=red!50!white] (c0) circle[radius=.05];
		\filldraw[very thin,draw=green!30!black,fill=green!50!white] (c1) circle[radius=.05];
		\filldraw[very thin,draw=blue!30!black,fill=blue!50!white] (c2) circle[radius=.05];
		\filldraw[very thin,draw=yellow!30!black,fill=yellow!50!white] (c3) circle[radius=.05];
	}{%
	}


	\draw[very thick,draw=blue] (r00) arc[start angle=\fpeval{\vdeg},end angle=\fpeval{90+\vdeg},radius=\vr] (r01) -- (r10) arc[start angle=\fpeval{90+\vdeg},end angle=\fpeval{180+\vdeg},radius=\vr] (r11) -- (r20) arc[start angle=\fpeval{180+\vdeg},end angle=\fpeval{270+\vdeg},radius=\vr] (r21) -- (r30) arc[start angle=\fpeval{270+\vdeg},end angle=\fpeval{360+\vdeg},radius=\vr] (r31) -- (r00);
}

% [step]{xmin}{xmax}{ymin}{ymax}
\newcommand{\drawcsys}[5][0]{
	\def\vstep{#1}
	\def\vxmin{#2}
	\def\vxmax{#3}
	\def\vymin{#4}
	\def\vymax{#5}

	\ifstrequal{#1}{0}{}{%
		\draw[help lines,step=\vstep] (\vxmin,\vymin) grid (\vxmax,\vymax);
	}
	\draw[->,thick] (\vxmin,0) -- (\vxmax,0) node[anchor=north] {$x$};
	\draw[->,thick] (0,\vymin) -- (0,\vymax) node[anchor=east]  {$y$};
}
