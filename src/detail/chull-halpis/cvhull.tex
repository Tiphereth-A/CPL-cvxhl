\begin{frame}{凸包}
	\label{cvh:ssec:cvhull}

	\begin{definition}[凸包]
		\label{cvh:def:cvhull}

		对于一个点集 \(P\), 其凸包即为能够覆盖 \(P\) 中所有点的凸多边形的交
	\end{definition}
\end{frame}


\begin{frame}[fragile]{算法 - Andrew 算法}
	\label{cvh:algo:andrew}

	\only<1>{常用的求凸包的算法有 Graham 算法和 Andrew 算法, 此处仅介绍 Andrew 算法}

	\only<2-4>{首先把所有点以横坐标为第一关键字, 纵坐标为第二关键字排序}

	\only<3-4>{显然排序后最小的元素和最大的元素一定在凸包上. 而且因为是凸多边形, 我们如果从一个点出发逆时针走, 轨迹总是``左拐''的, 一旦出现右拐, 就说明这一段不在凸包上. 因此我们可以用一个单调栈来维护上下凸壳}

	\only<4>{因为从左向右看, 上下凸壳所旋转的方向不同, 为了让单调栈起作用, 我们首先 \textbf{升序枚举} 求出下凸壳, 然后 \textbf{降序} 求出上凸壳}

	\only<5->{求凸壳时, 一旦发现即将进栈的点 (\(P\)) 和栈顶的两个点 (\(S_1,S_2\), 其中 \(S_1\) 为栈顶) 行进的方向向右旋转, 即叉积小于 \(0\): \(\overrightarrow{S_2S_1}\times \overrightarrow{S_1P}<0\), 则弹出栈顶, 回到上一步, 继续检测, 直到 \(\overrightarrow{S_2S_1}\times \overrightarrow{S_1P}\ge 0\) 或者栈内仅剩一个元素为止}

	\only<6->{通常情况下不需要保留位于凸包边上的点, 因此上面一段中 \(\overrightarrow{S_2S_1}\times \overrightarrow{S_1P}<0\) 这个条件中的``\(<\)''可以视情况改为 \(\le\), 同时后面一个条件应改为 \(>\)}

	\only<7->{\textbf{时间复杂度} \(O(n\log n)\), 其中 \(n\) 为点集大小}
\end{frame}


\begin{frame}[fragile,allowframebreaks]{实现 - Andrew 算法}
	\includecode[c++]{cvhull.cpp}
\end{frame}
