\begin{frame}{半平面交}
	\label{cvh:ssec:hpis}

	\begin{definition}[半平面]
		\label{cvh:def:halfplane}

		一条直线和其一侧的所有点, 如图 (\ref{fig:halfplane.tex})
	\end{definition}

	\includetikzimage{halfplane.tex}{半平面}
\end{frame}


\begin{frame}[fragile]{算法 - 排序增量算法 (S \& I 算法)}
	\label{cvh:algo:si}

	\only<1->{首先我们对所有的直线进行极角排序. 因为半平面交是一个凸多边形, 所以需要维护一个凸壳}

	\only<2->{因为后来加入的只可能会影响最开始加入的或最后加入的边, 只需要删除队首和队尾的元素, 所以需要维护一个单调队列

		我们遍历排好序了的直线, 并维护另一个交点数组, 当单队中元素超过 2 个时, 他们之间就会产生交点}

	\only<3->{对于当前直线, 如果上一个交点在这条直线表示的半平面交的 \textbf{异侧}, 那么上一条边就没有意义了}
\end{frame}


\begin{frame}[fragile]{算法 - 排序增量算法 (S \& I 算法)}
	\includetikzimage{halfplane2.tex}{单调队列}

	\only<1>{如图 (\ref{fig:halfplane2.tex}), 极角排序后, 遍历顺序应该是 \(\vec a\to\vec b\to\vec c\). 当 \(\vec a\) 和 \(\vec b\) 入队时, 在交点数组里会产生一个点 \(D\) (交点数组保存队列中相同下标的直线与前一直线的交点)}

	\only<2->{接下来枚举到 \(\vec c\) 时, 发现 \(D\) 在 \(\vec c\) 的右侧. 而因为 \textbf{产生} \(D\) \textbf{的直线的极角一定比} \(\vec c\) \textbf{要小}, 所以产生 \(D\) 的直线 (\(\vec b\)) 就对半平面交没有影响了}
\end{frame}


\begin{frame}[fragile]{算法 - 排序增量算法 (S \& I 算法)}
	还有一种可能的情况是快结束的时候, 新加入的直线会从队首开始造成影响, 如图 (\ref{fig:halfplane3.tex})

	\includetikzimage{halfplane3.tex}{队首影响}
\end{frame}


\begin{frame}[fragile]{算法 - 排序增量算法 (S \& I 算法)}
	\only<1->{仍然假设取直线左侧半平面. 加入直线 \(\vec f\) 之后, 第一个交点 \(G\) 就在 \(\vec f\) 的右侧, 我们把上面的判断标准逆过来看, 就知道此时应该删除直线 \(\vec a\), 也即 \textbf{队首} 的直线}

	\only<2->{最后用队首的直线排除一下队尾多余的直线. 因为队首的直线会被后面的约束, 而队尾的直线不会. 此时它们围成了一个环, 因此队首的直线就可以约束队尾的直线}

	\only<3->{如果半平面交是一个凸 \(n\) 边形, 最后在交点数组里会得到 \(n\) 个点. 我们再把它们首尾相连, 就是一个统一方向的多边形了}
\end{frame}


\begin{frame}[fragile]{注意}
	\only<1->{当出现一个可以把队列里的点全部弹出去的向量 (即所有队列里的点都在该向量的右侧), 则我们 \textbf{必须} 先处理队尾, 再处理队首}

	\only<2->{因此在循环中, 我们先枚举 \texttt{--r} 的部分, 再枚举 \texttt{++l} 的部分才不会错}
\end{frame}


\begin{frame}[fragile]{注意}
	\includetikzimage{halfplane4.tex}{}

	一般情况下, 我们在队列 (队列顺序为 \(\left\{\vec{u},\vec{v}\right\}\)) 后面加一条边 (向量 \(\vec w\)) , 会产生一个交点 \(N\), 缩小 \(\vec{v}\) 后面的范围
\end{frame}


\begin{frame}[fragile]{注意}
	\includetikzimage{halfplane5.tex}{}

	但是毕竟每次操作都是一般的, 因此可能会有把 \(M\) 点``挤出去''的情况
\end{frame}


\begin{frame}[fragile]{注意}
	\includetikzimage{halfplane6.tex}{}

	\only<1-2>{如果此时出现了向量 \(\vec a\), 使得 \(M\) 在 \(\vec a\) 的右侧, 那么 \(M\) 就要出队了. 此时如果从队首枚举 \texttt{++l}, 显然是扩大了范围}

	\only<2>{实际上 \(M\) 点是由 \(\vec u\) 和 \(\vec v\) 共同构成的, 因此需要考虑影响到现有进程的是 \(\vec u\) 还是 \(\vec v\). 而因为我们在极角排序后, 向量是逆时针顺序, 所以 \(\vec v\) 的影响要更大一些}

	\only<3->{如图 (\ref{fig:halfplane6.tex}), 如果 \(M\) 确认在 \(\vec a\) 的右侧, 那么此时 \(\vec v\) 的影响一定不会对半平面交的答案作出任何贡献}

	\only<4->{而我们排除队首的原因是 \textbf{当前向量的限制比队首向量要大}, 这个条件的前提是队列里有不止两个线段 (向量), 不然就会出现上面的情况, 所以一定要先排除队尾再排除队首}
\end{frame}


\begin{frame}[fragile,allowframebreaks]{实现 - 排序增量算法}
	\includecode[c++]{halfplane.cpp}
\end{frame}
